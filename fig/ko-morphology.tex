\begin{figure}[!htbp]
\begin{center}

\begin{minipage}[t]{\linewidth}

\hspace*{\fill}%
\begin{minipage}[t]{0.22\linewidth}
\centering
\vspace{0pt} % for alignment
\begin{minipage}[b]{\textwidth}
\adjincludegraphics[width=\textwidth, trim={{.0\width} {.0\width} {.5\width} {.5\width}}, clip]{img/knockout/morphology/wildtype/seed=1+title=channel_viz+treat=resource-even__channelsense-yes__nlev-two+update=8188+_data_hathash_hash=cb64cdf045bc6049+_script_fullcat_hash=7e789c981e3d0e4f+_source_hash=53a2252-clean+ext=}
{\textbf{(A)} Wild type}
% \label{fig:morphology-wt}
\end{minipage}
\end{minipage}%
\hfill
\begin{minipage}[t]{0.22\linewidth}
\centering
\vspace{0pt} % for alignment
\begin{minipage}[b]{\textwidth}
\adjincludegraphics[width=\textwidth, trim={{.0\width} {.0\width} {.5\width} {.5\width}}, clip]{img/knockout/morphology/knockout/seed=1+title=channel_viz+treat=resource-even__channelsense-yes__nlev-two+update=8188+_data_hathash_hash=9a4119947348e91d+_script_fullcat_hash=7e789c981e3d0e4f+_source_hash=53a2252-clean+ext=}
{\textbf{(B)} Messaging knockout}
% \label{fig:morphology-ko}
\end{minipage}
\end{minipage}%
\hspace*{\fill}

\hspace*{\fill}%
\begin{minipage}[t]{0.45\linewidth}
\centering
\vspace{0pt} % for alignment
\begin{minipage}[b]{\textwidth}
\adjincludegraphics[width=\textwidth]{img/knockout/morphology/title=group_shape+_data_hathash_hash=cb1733796dea778f+_script_fullcat_hash=68cf35a1759c64ac+_source_hash=53a2252-clean+ext=}
{\textbf{(C)} Distribution of L0 same-hereditary-group neighbor counts.}
% \label{fig:morphology-shape}
\end{minipage}
\end{minipage}%
\hspace*{\fill}%
\hspace*{\fill}%
\begin{minipage}[t]{0.45\linewidth}
\centering
\vspace{0pt} % for alignment
\begin{minipage}[b]{\textwidth}
\adjincludegraphics[width=\textwidth]{img/knockout/morphology/title=group_perimeter_area+_data_hathash_hash=b02d4442d68976b7+_script_fullcat_hash=4198d7d7c0b9f172+_source_hash=53a2252-clean+ext=}
{\textbf{(D)} L0 hereditary group stringiness measure versus group sizes.}
% \label{fig:morphology-factor}
\end{minipage}
\end{minipage}%
\hspace*{\fill}

\end{minipage}

\caption{
Comparison of a wild type strain evolved under the ``Nested-Even'' treatment with stringy L0 hereditary groups and the corresponding intracellular-messaging knockout strain.
Subfigures \textbf{(A)} and \textbf{(B)} visualize hereditary group layouts;
color hue denotes and black borders divide L1 hereditary groups while color saturation denotes and white borders divide L0 hereditary groups.
Smaller, thinner, and more elongated L0 groups can be seen in the wild type strain than in the knockout strain.
Subfigures \textbf{(C)} and \textbf{(D)} quantify the morphological effect of the intracellular-messaging knockout.
In the formula for Shape Factor given in Subfigure \textbf{(C)}, $P$ refers to group perimeter and $A$ refers to group area.
Error bars indicate 95\% confidence.
View an animation of the wild type strain at \url{https://hopth.ru/q}.
View the wild type strain in a live in-browser simulation at \url{https://hopth.ru/f}.
}
\label{fig:ko-morphology}
\end{center}
\end{figure}
