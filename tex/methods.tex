\section{Methods}

We performed simulations in which cells evolved open-ended behaviors to make decisions about resource sharing, reproductive timing, and apoptosis.
We will first describe the environment and hereditary grouping system cells evolved under and then describe the behavior-control system cells used.

\subsection{Environment and Hereditary Groups}

Cells occupy individual tiles on a toroidal grid.
Over discrete time steps (``updates''), cells can collect a resource.
Once sufficient resource has been accrued, cells may pay one unit of resource to place a daughter cell on an adjoining tile of the toroidal grid (i.e., reproduce), replacing any existing cell already there.
Collected resource decays at a rate of 0.1\% per update, incentivizing its quick use.

Cells accrue resource via a cooperative resource-collection process conducted by explicitly-registered hereditary groups.
As cells reproduce, they can choose to keep offspring in the parent's hereditary group or expel offspring to found a new hereditary group.
These decisions affect the spatial layout of these hereditary groups and, in turn, affect individual cells' resource-collection rate.
Medium-sized, circular hereditary groups tend to collect resource at a greater per-cell rate than large, small, or irregularly-shaped groups.
To promote group turnover, we counteract the advantage of established hereditary groups with a simple aging scheme.
As hereditary groups age over elapsed updates and somatic generations, their constituent cells expressly lose the ability regenerate somatic tissue and then, soon after, to collect resource.

A complete description of the mechanisms behind these collective resource-collection and group aging mechanisms appears in supplementary sections \ref{sup:resource_collection_process} and \ref{sup:channel_group_life_cycle}.

Because hereditary groups arise through inheritance, they signify a kin relationship in addition to a potentially functionally cooperative relationship.
In this work, we screen for fraternal transitions in individuality with respect to these hereditary groups by evaluating three characteristic traits of higher-level organisms: resource sharing, reproductive division of labor, and apoptosis.

\subsection{Cell-Level Organisms}

Our experiments use cell-level digital organisms controlled by genetic programs subject to mutations and implicit selective pressures.
We employ the SignalGP representation, which expresses function-like modules of code in response to internal signals or external stimuli (akin to gene expression).
This event-driven paradigm facilitates the evolution of dynamic interactions between digital organisms and their environment (including other organisms) \citep{lalejini2018evolving}.

Previous work evolving digital organisms in grid-based problem domains has relied on a single computational agent that designates a direction to act in via an explicit cardinal ``facing'' \citep{goldsby2014evolutionary, goldsby2018serendipitous, grabowski2010early, biswas2014causes, lalejini2018evolving}.
We introduce novel methodology to facilitate the evolution of directionally-symmetric phenotypes.
In this work, each cell instantiates four copies of the SignalGP hardware: one facing each cardinal direction.
These hardware instances all execute the same SignalGP program and may coordinate via internal signals, but are otherwise decoupled.
Supplementary Figure \ref{fig:dishtinygp-cartoon} overviews the configuration of the four SignalGP instances that constitute a single cell.

\subsection{Surveyed Evolutionary Conditions}

To broaden our exploration of possible evolved multicellular behaviors in this system, we surveyed several evolutionary conditions.

In one manipulation, we explored the effect of structuring hereditary groups, such that parent cells can choose to keep offspring in their same sub-group, in just the same full group, or expel them entirely to start a new group.
Cells can independently mediate their behavior based on the level of the group with which they are interacting.

In a second manipulation, we explored the importance of explicitly selecting for medium-sized groups (as had been needed to maximize resource collection) by removing this incentive.
Instead, they system distributed resource at a uniform per-cell rate.

We combined these two manipulations to yield four surveyed conditions:
\begin{enumerate}
\item ``Flat-Even'': One hereditary level (flat) with uniform resource inflow (even). In-browser simulation: \url{https://mmore500.com/hopto/i},
\item ``Flat-Wave'': One hereditary level (flat) with group-mediated resource collection (wave); In-browser simulation: \url{https://mmore500.com/hopto/j}),
\item ``Nested-Even'': Two hierarchically-nested hereditary levels (nested) with uniform resource inflow (even). In-browser simulation: \url{https://mmore500.com/hopto/k},
\item ``Nested-Wave'': Two hierarchically-nested hereditary group levels (nested) with group-mediated resource collection (wave). In-browser simulation: \url{https://mmore500.com/hopto/l}.
\end{enumerate}
