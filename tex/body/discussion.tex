\section{Discussion}

% In previous work exploring the DISHTINY platform, we used simple organisms that evolve parameters for a set of manually designed strategies to demonstrate that DISHTINY selects for genotypes that exhibit high-level individuality \cite{moreno2019toward}.

In this work, we selected for fraternal transitions in individuality among digital organisms controlled by genetic programs.
Because --- unlike previous work \citep{goldsby2012task, goldsby2014evolutionary} --- we provided no experimentally prescribed mechanism for collective reproduction, we observed the emergence of several distinct life histories.
Evolved strategies exhibited intercellular communication, coordination, and differentiation.
These included endowment of offspring propagule groups, asymmetrical intra-group resource sharing, asymmetrical inter-group relationships, morphological patterning, gene-regulation mediated life cycles, and adaptive apoptosis.

Across treatments, we observed resource-sharing and reproductive cooperation among registered kin groups.
These outcomes arose even in treatments where registered kin groups lacked functional significance (i.e., resource was distributed evenly), suggesting that reliable kin recognition alone might be sufficient to observe aspects of fraternal collectivism evolve in systems where population members compete antagonistically for limited space or resources and spatial mixing is low.
In addition to their functional consequences, perhaps the role of physical mechanisms such as cell attachment simply as a kin recognition tool might merit consideration.

In future work, we are eager to undertake experiments investigating open questions pertaining to major evolutionary transitions such as the role of pre-existing phenotypic plasticity \citep{clune2007investigating, lalejini2016evolutionary}, pre-existing environmental interactions, pre-existing reproductive division of labor, and how transitions relate to increases in organizational \citep{goldsby2012task}, structural, and functional \citep{goldsby2014evolutionary} complexity.
Expanding the scope of our existing work to directly study evolutionary dynamics and evolutionary histories will be crucial to such efforts.

In particular, we plan to investigate mechanisms to evolve greater collective sophistication among agents.
The modular design of SignalGP lends itself to the possibility of exploring sexual recombination.
We are interested in exploring extensions to allow cell groups to develop neural and vascular networks \citep{Moreno_Ofria_2020}.
We hypothesize that selective pressures related to intra-group coordination and inter-group conflict might spur developmental and structural infrastructure that could be co-opted to evolve agents proficient at unrelated tasks like navigation, game-playing, or reinforcement learning.

Unfortunately, however, experiments with multicellularity are specially constrained by a fundamental limitation of digital evolution research: processing power \citep{Moreno_2020}.
This limitation, which commonly manifests as smaller population sizes compared to natural populations \citep{liard2018complexity}, only compounds when the unit of selection shifts to computationally expensive groups of dozens or hundreds of component individuals.
Ongoing work with DISHITNY is testing approaches to harness increasingly abundant parallel processing power for digital evolution simulation \citep{moreno2021conduit}.
The spatial, distributed nature of our approach potentially affords a route to achieve large-scale digital multicellularity experiments consisting of millions, instead of thousands, of cells via high-performance parallel computing.
