\begin{abstract}

\section{}
Evolutionary transitions occur when previously-independent replicating entities unite to form more complex individuals.
Such transitions have profoundly shaped natural evolutionary history and occur in two forms: fraternal transitions involve lower-level entities that are kin (e.g., transitions to multicellularity or to eusocial colonies), while egalitarian transitions involve unrelated individuals (e.g., the origins of mitochondria).
The necessary conditions and evolutionary mechanisms for these transitions to arise continue to be fruitful targets of scientific interest.
Here, we examine a range of fraternal transitions in populations of open-ended self-replicating computer programs.
These digital cells were allowed to form and replicate kin groups by selectively adjoining or expelling daughter cells.
The capability to recognize kin-group membership enabled preferential communication and cooperation between cells.
We repeatedly observed group-level traits that are characteristic of a fraternal transition.
These included reproductive division of labor, resource sharing within kin groups, resource investment in offspring groups, asymmetrical behaviors mediated by messaging, morphological patterning, and adaptive apoptosis.
We report eight case studies from replicates where transitions occurred and explore the diverse range of adaptive evolved multicellular strategies.

\tiny
 \keyFont{ \section{Keywords:} digital evolution, artificial life, major transitions in evolution, multicellularity, evolution} %All article types: you may provide up to 8 keywords; at least 5 are mandatory.
\end{abstract}
