\section{Introduction}

An evolutionary transition in individuality is an event where independently replicating entities unite to replicate as a single, higher-level individual \citep{smith1997major}.
These transitions are understood as essential to natural history's remarkable record of complexification and diversification \citep{smith1997major}.
Likewise, researchers studying open-ended evolution in artificial life systems have focused on transitions in individuality as a mechanism that is missing in digital systems, but necessary for achieving the evolution of complexity and diversity that we witness in nature \citep{taylor2016open, banzhaf2016defining}.

Here, we focus on \textit{fraternal} evolutionary transitions in individuality, in which the higher-level replicating entity is derived from the combination of cooperating kin that have entwined their long-term fates \citep{west2015major}.
Multicellular organisms and eusocial insect colonies exemplify this phenomenon \citep{smith1997major} given that both are sustained and propagated through the cooperation of lower-level kin.
Although not our focus here, egalitarian transitions --- events in which non-kin unite, such as the genesis of mitochondria by symbiosis of free-living prokaryotes and eukaryotes \citep{smith1997major} --- also constitute essential episodes in natural history.

In nature, major transitions occur rarely and over vast time scales, making them challenging to study.
Recent work in experimental evolution \citep{ratcliff2014experimental, ratcliff2015origins, gulli2019evolution, koschwanez2013improved}, mechanistic modeling \citep{hanschen2015evolutionary, staps2019emergence}, and digital evolution \citep{goldsby2012task, goldsby2014evolutionary} complements traditional post hoc approaches focused on characterizing the record of natural history.
These systems each instantiate the evolutionary transition process, allowing targeted manipulations to test hypotheses about the requisites, mechanisms, and evolutionary consequences of fraternal transitions.
Digital evolution, occupying a sort of middle ground between wet work and mechanistic modeling, offers a unique conjunction of experimental capabilities that complements work in both of those disciplines.
Like modeling, digital evolution affords rapid generational turnover, complete observability (every event in a digital system can be tracked), and complete manipulability (every event in a digital system can can be arbitrarily altered).
However, as with \textit{in vivo} experimental evolution, digital evolution systems can exhibit rich evolutionary dynamics stemming from complex, rugged fitness landscapes \citep{labar2017evolution} and sophisticated agent behaviors \citep{grabowski2013case}.

Our work here follows closely in the intellectual vein of Goldsby's deme-based digital evolution experiments \citep{goldsby2012task, goldsby2014evolutionary}.
In her studies, high-level organisms exist as a group of cells within a segregated, fixed-size subspace.
High-level organisms that must compete for a limited number of subspace slots.
Individual cells that comprise an organism are controlled by heritable computer programs that allow them to self-replicate, interact with their environment, and communicate with neighboring cells.

Goldsby's work defines two modes of cellular reproduction: tissue accretion and offspring generation.
Within a group, cells undergo tissue accretion, whereby a cell copies itself into a neighboring position in its subspace.
In the latter, a population slot is cleared to make space for a daughter organism then seeded with a single daughter cell from the parent organism.

Cells grow freely within an organism, but fecundity depends on the collective profile of computational tasks (usually mathematical functions) performed within the organism.
When an organism accumulates sufficient resource, a randomly chosen subspace is cleared and a single cell from the replicating organism is used as a propagule to seed the new organism.
This setup mirrors the dynamics of biological multicellularity, in which cell proliferation may either grow an existing multicellular body or found a new multicellular organism.

Here, we take several steps to develop a computational environment that removes the enforcement and rigid regulation of multiple organismal levels.
Specifically, we remove the explicitly segregated subspaces and we let cell interact with each other more freely.
We demonstrate the emergence of multicellularity where each organism manages its own spatial distribution and reproductive process.
This spatially-unified approach enables more nuanced interactions among organisms, albeit at the cost of substantially more complicated analyses.
Instead of a single explicit interface to mediate interactions among high-level organisms, such interactions must emerge via many cell-cell interfaces.
Novelty can occur in terms of interactions among competitors, among organism-level kin, or even within the building blocks that make up hierarchical individuality.
Experimentally studying fraternal transitions in a digital system where key processes (reproductive, developmental, homeostatic, and social) occur implicitly within a unified framework can provide unique insights into nature.

We do provide some framework to facilitate fraternal transitions in individuality by allowing cells to readily designate distinct hereditary groups.
Offspring cells may either remain part their parent's hereditary group or found a new group.
Cells can recognize group members, thus allowing targeted communication and resource sharing with kin.
We reward cells for performing tasks that require proper timing such that they must coordinate to be successful.
As such, cells that cooperate will have an advantage on those tasks and if they are also part of a hereditary group they will increase their inclusive fitness.
In previous work introducing the DISHTINY (DIStributed Hierarchical Transitions in IndividualitY) framework we evolved parameters for manually-designed cell-level strategies to explore fraternal transitions in individuality \citep{moreno2019toward}.
Here, we build on this prior work with DISHTINY to incorporate a more open-ended event-driven genetic programming representation called SignalGP, which was designed to facilitate dynamic interactions among agents and between agents and their environment \citep{lalejini2018evolving}.
As expected, we see a far more diverse set of behaviors and strategies arise.
Here, we report case studies of notable multicellular phenotypes that evolved via this more dynamic genetic programming underpinning.
We see these anecdotal characterizations as a precursory step toward hypothesis-driven work contributing to open questions about fraternal transitions in individuality.
